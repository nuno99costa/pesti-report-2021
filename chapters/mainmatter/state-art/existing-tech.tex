%!TEX root = ../../../main.tex

\subsection{Existing Technologies}

\acrfull{hpo} is an important step in the machine learning development pipeline. This is usually implemented into other frameworks (as part of a complete pipeline framework). We pay special attention to existing frameworks that are model agnostic, written in \acrfull{python} and are able to treat the objective function as a \textit{black box}.

We also also specify important software tools or protocols used.

\subsubsection{Hyper Parameter Optimization}

\paragraph{Open-source Software}

Due to the academic nature of the development of \acrshort{hpo} frameworks, there are a plethora of frameworks, with different optimization techniques, goals and interfaces. We present some open source frameworks that introduce specific features or optimization techniques, are industry standard as well as some commercial offerings.

\subparagraph{Determined} This is an open-source deep learning training platform \parencite{determined-ai}. It allows for distributed training using a custom Horovod-based framework \parencite{alex2018horovod} as well as features to better utilize computational resources (smart scheduling). It's \acrshort{hpo} interface supports ASHA \parencite{li2018massively}, grid search, random search and population-based training. This framework is specifically focused on deep learning, which makes it harder to use with black-box functions.

\subparagraph{HpBandSter} This \acrshort{python} package implements a distributed HyperBand algorithm, allowing for parallelization of the \acrshort{hpo} \parencite{hpbandster}. It also implements BOHB, which combines bayesian optimization and HyperBand \parencite{pmlr-v80-falkner18a}.

\subparagraph{hyperopt} This \acrshort{python} library is intended to be used for serial and parallel \acrshort{hpo} \parencite{hyperopt}. It implements random search and tree parzen estimators. It's parallelization feature can be executed using Apache Spark or MongoDB.

\subparagraph{scikit-learn} This framework is one of the most popular frameworks for machine learning in \acrshort{python}. It implements a full development pipeline for machine learning, including \acrshort{hpo} \parencite{scikit-learn}. It implements grid search, random search and successive halving as default available optimization algorithms. It also has community developed add-on algorithms, such as Auto-sklearn \parencite{auto-sklearn} and scikit-optimize \parencite{scikit-optimize}.

\subparagraph{Tune} This \acrshort{python} library is part of Ray \parencite{ray}, an open source framework that provides a simple, universal API for building distributed applications \parencite{liaw2018tune}. It supports distributed computing and multiple \acrshort{gpu}s per computing node. It implements random search, grid search, a number of bayesian optimization algorithms, Tree-Parzen estimators, HyperBand, gradient-free optimization and BOHB.

\paragraph{Commercial Services} 

As the machine learning field grows, companies are trying to create services tailored to machine learning focused customers.

\subparagraph{Amazon Sagemaker} This cloud \acrshort{machinel} platform, developed by Amazon, is a complete suite of tools for creating, training and deploying \acrshort{machinel} in the cloud \parencite{sagemaker}. This platform implements \acrshort{hpo} algorithms such as random search and bayesian optimization.

\subparagraph{Google HyperTune} This training platform is part of Google's AI Platform \parencite{ghypertune}. It supports bayesian optimization, grid search and random search.   

\subsubsection{Virtualization}

As a part of the development of the software solution, it was necessary to implement a virtualization mechanism to harmonize the development environment (Windows) and the target \acrfull{os} of the Ray framework (Linux). These techniques can be separated on the level at which they operate - hardware virtualization, desktop virtualization and \acrshort{os}-level virtualization. \todo{Should I add this to the related works section?} Here we focus our research on \acrshort{os}-level virtualization, as it was at this level (where we are able to run another \acrshort{os}) that the solution was developed on.

\acrshort{os}-level virtualization is a technology that partitions the operating system to create multiple isolated \acrfull{vm} \parencite{10.5555/1571423}. \todo{Should I refer advantages of os-level virtualization?}

\paragraph{Chroot} This Unix operation was initially introduced in 1979. Chroot can be considered to be a rudimentary form of virtualization, as it changes the apparent root directory for the current running process and any software running in this environment cannot access files outside the designated root directory, effectively limiting it's access to a directory tree, while still allowing acess to other resources \parencite{foundation_2017}.

\paragraph{Docker/Moby} Docker is a set of \acrfull{paas} products that use \acrshort{os}-level virtualization to package software into containers. These tools are open sourced through the Moby Project \parencite{moby_oro}. Docker is one of the most used container runtime engines, with ample documentation, examples and support from the community \parencite[p. 20]{stateofcontainers}.


\paragraph{Kubernetes} This system, also known as K8s, is an open source system for managing containerized applications across multiple hosts. It provides basic mechanisms for deployment, maintenance, and scaling of applications \parencite{kubernetes}. This system is one of the most used in the industry (both as a self-managed solution as well as prebuilt services such as \acrfull{aks}) \parencite{stateofcontainers}.

\paragraph{VMware Workstation} This virtualization software is one of the most used with regards to hardware level optimization. Although this type of virtualization allows for better control over hardware (allowing for their complete simulation), it increases the overhead of running any software service \parencite{bauer_2019}.

\todo[color=green]{Professor:\\Should I mention web APIs (used them to connect to CAM2 software), Python modules (the way I use Ray Tune)?}