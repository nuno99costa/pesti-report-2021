%!TEX root = ../../../main.tex

\subsection{Functional and Non-functional Requirements}

\todo[color=magenta]{Hooshiar:\\Please take a look here and confirm that these requirements are correct and complete.}

We can arrange the defined requirements based on their typology, regarding whether they are directly impacting the project (e.g.\ features) or if they are used to specify the criteria for the success of the system (e.g.\ failure tolerance, uptime and others). With this, we defined the requirements for this project based on these definitions.

\subsubsection{Functional Requirements}

Initially four use cases where defined:

\todo[color=green]{Professor:\\Is this sufficiently developed? Do I need any more "use cases"? Due to the very direct approach to the project (optimize this function) there isn't a lot of uses cases.}

\begin{itemize}
	\item A privileged developer can integrate new \acrshort{hpo} algorithms into the framework.
	\item A developer can select which optimization algorithms will be ran.
	\item A developer can define all necessary information to allow the optimization pipeline to run (hyper parameter definitions, objective function instantiation, etc.).
	\item A developer can obtain information during and at the end of the optimization process regarding said process.
\end{itemize}

Here we define a privileged developer as a developer who is connected to the development of the framework.

\subsubsection{Non-functional Requirements}

\paragraph{Usability}

\paragraph{Reliability}

\paragraph{Performance}

\paragraph{Supportability}

\paragraph{Implementation Constraints} 

These constraints define how the project is developed. They can, for instance, to define used coding standards and conventions.

\begin{itemize}
	\item The developed application must be implemented in Python.
	\item Execution of the application must be possible using the standard company issued laptop.
\end{itemize}

\paragraph{Interface Constraints} 

These define how the communication between various components of the project is executed. With this, we can define constraints based on the communication between the CAM2\textsuperscript{\textregistered} software and the developed project:

\begin{itemize}
	\item The optimization pipeline must allow communication with the CAM2\textsuperscript{\textregistered} integrated web server.
	\item Said communication must be executed through HTTP, using pre existing endpoints.
\end{itemize}


