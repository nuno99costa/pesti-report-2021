%!TEX root = ../../../main.tex

\subsection{Requirements}

\todo[color=magenta]{Hooshiar:\\Please take a look here and confirm that these requirements are correct and complete.}

We can arrange the defined requirements using the \acrfull{furps} model \parencite{furps}.

\begin{itemize}
	\item A privileged developer can integrate new \acrshort{hpo} algorithms into the framework.
	\item A developer can select which optimization algorithms will be ran.
	\item A developer can define all necessary information to allow the optimization pipeline to run (hyper parameter definitions, objective function instantiation, etc.).
	\item A developer can obtain information during and at the end of the optimization process regarding said process.
\end{itemize}

Here we define a privileged developer as a developer who is connected to the development of the framework.

\subsubsection{Functionality}

These constraints define the capability, reusability and security of a given project or application. In the scope of this project we can define the following requirements:

\begin{itemize}
	\item 
\end{itemize}

\subsubsection{Usability}

These constraints define how the interaction between a user and the program occurs, as well as, for example, the available documentation. In the scope of this project we can define the following requirements:

\begin{itemize}
	\item 
\end{itemize}

\subsubsection{Reliability}

These constraints define important factors such as the availability and robustness of the system, as well as it's accuracy and stability. In the scope of this project we can define the following requirements:

\begin{itemize}
	\item 
\end{itemize}

\subsubsection{Performance}

These constraints define what the expected performance of a given application is, using metrics such as speed, efficiency, resources consumption and others. In the scope of this project we can define the following requirements:

\begin{itemize}
	\item 
\end{itemize}

\subsubsection{Supportability}

These constraints define how easily the developed project is supported and used by others. To do this, we can define goals regarding testability, modifiability, installability and others. In the scope of this project we can define the following requirements:

\begin{itemize}
	\item 
\end{itemize}

\subsubsection{Design Constraints}

These constraints define the design of a system and may include programming languages, software processes, tool usage and others. In the scope of this project we can define the following requirements:

\begin{itemize}
	\item The developed application must be implemented in Python.
\end{itemize}

\subsubsection{Implementation Constraints} 

These constraints define how the project is developed and may include required standards, resources limitations and others. In the scope of this project we can define the following requirements:

\begin{itemize}
	\item Execution of the application must be possible using the standard company issued laptop.
	\item The developed application must be able to execute in a Windows environment.
\end{itemize}

\subsubsection{Interface Constraints} 

These constraints define how the communication between various components of the project is executed. Within the scope of this project, we can define constraints based on the communication between the CAM2\textsuperscript{\textregistered} software and the developed project:

\begin{itemize}
	\item The optimization pipeline must allow communication with the CAM2\textsuperscript{\textregistered} integrated web server.
	\item Said communication must be executed through HTTP, using pre existing endpoints.
\end{itemize}

\subsubsection{Physical Constraints}

These constraints define the physical limitations for the system. Such constraints can be material, shape or weight specifications when referring to a hardware project but also refer to, for example, network requirements or topology when referring to software projects. In the scope of this project we can define the following requirements:

\begin{itemize}
	\item 
\end{itemize}
