%!TEX root = ../../../main.tex

\subsection{Problem Description}

FARO Technologies\textsuperscript{\textregistered} is increasingly investing in new machine learning based features in it's software suite. As such, there is a need to create effective tools for algorithm training that fit the specific algorithm and integrate into the pre-existing workflow and continuous integration pipelines, which this project focuses on.

\subsubsection{Objectives}

The objective of this internship was to create a solution for optimizing parameters in the development process and deployment of the software solutions developed by \faro. The main goals for this internship were:

\begin{itemize}
	\item Develop 3 algorithm training solutions
	\item Develop meaningful unit tests for these solutions
	\item Integrate the solutions into pre-existing software
	\item Demonstrate value by solving real world issues in CAM2\textsuperscript{\textregistered} software with the developed solutions
\end{itemize}

\subsubsection{Approach}

This project was developed by 1 student, working remotely for 4 months. Initially a simple solution was developed, in order for the student to get familiarized with the development process and tools used by the company. Afterwards, 2 more solutions were developed, as well as unit testing and integrations with other software developed by the company.\todo{What is the approach used (citation)?}

\subsubsection{Contributions}

The beneficiaries from the results of this internship are the customers and developers at \faro. These customers apply \faro products and services in a wide range of industries (from assembly line machine calibration \parencite{faro_man_eq_align}, forensic analysis \parencite{faro_forensic} to construction quality control \parencite{faro_construction}) where these solutions impact the reliability, quality control procedures and legal compliance of factories, buildings and legal work. As such, any increase in productivity (through better accuracy, speed or new features) made possible by an implementation of machine learning affects a large number of companies, industries, supply chains and projects significantly. Therefore, it is extremely important that the machine learning algorithms used are properly trained, so that they're default parameters are correct and perform well under real conditions. This project will help further this goal and speed up future machine learning based features introduced by \faro.

Developers are also affected, as the developed tooling can and likely will be used for future work when optimizing function parameters through the entire code base at \faro.

\subsubsection{Planning}

This project was overseen by the Research Department at Antares Lda., which meet daily as part of the Scrum development model used, where each member discussed their progress and next steps towards their specific objectives. 

Due to the time limitations imposed by the internship, goals were set at the beginning of the internship as part of a 1/15/45/90/120 internship overview plan. These goals were used to guide the work and evaluate progress throughout the internship.

As part of the development process, all functionally significant development process artifacts were logged using Git as a version control system.

All communication used Microsoft Teams\todo{Will I be using JIRA or any other tool besides Teams and Git (besides any framework that is used for coding)?}, due to the pandemic situation during the internship.