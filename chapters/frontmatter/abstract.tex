%!TEX root = ../../main.tex

\section*{Abstract}

Nowadays, \acrfull{machinel} is a growing field of study academically and a growing part of software development. \acrshort{machinel} provides accurate, fast results, based on mathematical concepts, with very little input from developers. However, in order to generate good models for wide spread usage, it is important to tune whichever model is being used, through \acrfull{hpo}.

This project consists of applying the concept of \acrshort{hpo} to a non-\acrshort{machinel} function (\acrfull{pca}) contained within a pre-existing software package (\acrshort{cam2}). To do this, we use a Python framework for \acrshort{hpo} of \acrshort{machinel} models (Ray Tune), applying it to a standard function, using an \acrshort{api} made available by the software package. We also implemented a data visualization tool (Weight and Biases), in order to visualize the obtained data through graphs.

The developed solution fulfils the initial requirements set out by the internship supervisors and provides insight into the analysed function, as well as providing future improvement options. While developing the solution, the author has learned new concepts and methodologies, which were put into practice throughout the internship, with success.

\begin{table}[ht]
\begin{tabular}{ll}
\textbf{Keywords (Subject):}      & Machine Learning, Software Optimization, 3D Software \\
\textbf{Keywords (Technologies):} & Python, REST APIs, Hyper Parameter Optimization, Data \\ 								   & Visualization
\end{tabular}
\end{table}